\documentclass[12pt]{article}
%\usepackage{yfonts}
\usepackage{mathrsfs}
\usepackage{amssymb}
\usepackage{amsmath}
\usepackage{graphicx}
\usepackage{subfigure}
\usepackage[left=2.2cm,right=2.2cm,top=3cm,bottom=2.1cm]{geometry}
\begin{document}

\begin{center}
\LARGE{\sc{Mec\'anica Anal\'itica}}\\
\vspace{0.5cm}
\large{\bf{Extracr\'edito 1}}\\
\vspace{0.5cm}
\scriptsize{Universidad de Chile, Facultad de Ciencias,}\\
\scriptsize{Departamento de F\'isica, Santiago, Chile}\\
\scriptsize{Jueves 22 de Diciembre, 2011}

\vspace{0.5cm}
\end{center}

\begin{flushleft}
\qquad {\bf{\large{Nombre:} }}{\sc{Alejandro Bernardin S.}}
\end{flushleft}

\section{ Extracr\'edito: Simulaci\'on problema de 3 cuerpos.}

Este documento tiene como finalidad explicar la simulaci\'on del
problema de 3 cuerpos adjunta,
tanto en su c\'odigo fuente (adem\'as de los comentarios dentro del
c\'odigo) como las ecuaciones ocupadas en ella.
Adem\'as se hace una peque\~na referencia a su utilizaci\'on.

\subsection {Ecuaciones}

Las ecuaciones ocupadas para esta simulaci\'on han sido las ecuaciones
de gravitaci\'on desarrolladas por Newton, o sea

$$U=\frac{m_im_j}{r_{ij}}$$
$$F=\frac{m_im_j}{r_{ij}^2}$$

\noindent las cuales al calcularlas para cada part\'icula resultan

$$\Delta U_j=\sum_{i=1}^n\frac{m_im_j}{r_{ij}}$$
$$\Delta F_j=\sum_{i=1}^n\frac{m_im_j}{r_{ij}^2}$$

\noindent siendo la particula $j$ la afectada tanto por el potencial
$\Delta U_j$ como
por la fuerza $\Delta F_j$ en un tiempo determinado $\tau$.\\

Debido a que nuestro sistema posee m\'as de 2 cuerpos, no posee soluci\'on
anal\'itica, por ende debemos integrar num\'ericamente nuestras
ecuaciones, para lo cual hemos utilizado el algoritmo \textbf{velocity
verlet}, el cual dice

$$\vec r_{n+1}=\vec r_n+\tau\vec v_n+\frac12\tau^2\vec a(\vec r_n)$$
$$\vec v_{n+1}=\vec v_n+\frac{\tau}{2}[\vec a_n(\vec r_n)+\vec a(\vec
r_{n+1})]$$

\noindent con el cual  hemos obtenido la posici\'on de cada part\'icula en
nuestro sistema.

\subsection {C\'odigo Fuente}

El programa principal viene dado por el archivo 3cuerpos.cc el cual se
complementa con las clases Part.cc, Cont.cc, azar.cc y Vect6.cc.
Analizaremos brevemente cada archivo:

\subsubsection{3cuerpos.cc} Este es el archivo principal, el cual
contiene el par\'ametro de entrada que representa la cantidad de
cuerpos del sistema, posee el integrador \textbf{velocity verlet},
posee las condiciones de borde periodicas, adem\'as
que es el encargado tanto de leer archivos externos como de escribir
en ellos.\\

\subsubsection{Cont.cc} Este archivo es el contenedor de las
part\'iculas, esta encargado de calcular la energ\'ia potencial
entre ellas as\'i como la fuerza.\\

\subsubsection{Part.cc} Este archivo es el que crea y setea las
part\'iculas, posee toda la informaci\'on de ellas (posici\'on,
velocidad y masa).\\

\subsubsection{Vect6.cc} Este archivo crea un vector de 6 dimensiones
utilizado para cosas muy espec\'ificas dentro del programa.\\

\subsubsection{azar.cc} Este archivo genera n\'umeros al azar.\\

\subsection {Forma de uso} El archivo ejecutable viene dado por el
archivo binario
\textbf{3cuerpos}, compilado para procesador de 64 bits. En caso de no
tener m\'aquina de 64 bits, compilar nuevamente ejecutando el comando
\textbf{make}. Al ejecutar el programa, este leer\'a autom\'aticamente
el archivo \textbf{entrada.xyz}, el cual puede ser modificado para ingresar
distintas condiciones iniciales a nuestra simulaci\'on de 3 cuerpos.\\

En caso de querer ingresar m\'as part\'iculas en el archivo entrada.xyz, solo es necesario cambiar la
variable $p$ y $archiv$ por la cantidad de part\'iculas deseadas en el archivo
3cuerpos.cc y compilar nuevamente, ya que el programa est\'a
capacitado para trabajar con $n$ part\'iculas.\\ 

El archivo de salida es del tipo .xyz, que puede ser visualizado en el
programa Xmakemol (se recomienda visualizar con una velocidad de 19).
El archivo entrada.xyz, que viene por defecto, nos muestra dos
planetas orbitando alrededor de otro 500 veces m\'as grande que ambos.
Se puede ejecutar autom\'aticamente Xmakemol ejecutando el archivo
\textbf{visualizar.bash}.

\subsection{Copyleft} El presente programa es de completa autor\'ia
del autor de este extracr\'edito (exceptuando la clase azar.cc, creado
por J.Rogan) y puede ser difundido, modificado, analizado y d\'arsele cualquier
uso que el usuario final estime conveniente. 


\end{document}
